% Use the article class and an 11pt letter:
\documentclass[11pt]{article}

% Use A4 paper and set margins.  Comment this line out (prepend
% a %) if the geometry package is missing on your system:
\usepackage[a4paper, top=2.5cm, left=2.5cm, bottom=2.5cm, right=2.5cm]{geometry}

% \usepackage{parskip}  % Comment this in in order to delineate paragraphs with an empty line, rather than an indentation


% Setup the title page:
\title{A very simple document}
\author{Marc van der Sluys\\
HAN University of Applied Sciences\\
Arnhem, The Netherlands}


% Above is the preamble, below the actual document.
\begin{document}

% Create a title:
\maketitle

% Create a ToC:
\tableofcontents

% Continue on a new page:
% \newpage

% A section:
\section{Introduction}

This is one of the smallest \LaTeX\ documents ever, to see whether you can compile this to a PDF.  

\section{Compiling your document}

In order to generate the PDF file from the \LaTeX\ source file \texttt{SimpleDocument.tex}, you need to
\emph{compile} the source.  People typically use TexWorks (\texttt{tug.org/texworks)}, the combination of
MikTeX \texttt{miktex.org} and TexStudio \texttt{texstudio.org} or the website \texttt{www.overleaf.com}.

\section{Style}

\subsection{Paragraphs}

Here is some text that is hopefully long enough to create a small paragraph.  If the text is \emph{not} long
enough, I'll add more.
If it is \emph{still} not long enough, I'll add even more.
If it is \emph{still} not long enough, I'll add even more.
If it is \textbf{still} not long enough, I'll add even more.
If it is \emph{still} not long enough, I'll add even more.
If it is \emph{still} not long enough, I'll add even more.
If it is \emph{still} not long enough, I'll add even more.

Note that hard returns in the code do nothing.  However, an empty line creates a new paragraph.  Note that the
default paragraph style uses indentation.  If you use the \texttt{parskip} package in the preamble
(``header'') of the source file, it will turn into an empty line.  You may need to have to install the package
first.


\subsection{Equations}

Creating an equation is easy, as illustrated by Equation~\ref{eq:example}.
\begin{equation}
  -e^{i \pi} = 1
  \label{eq:example}
\end{equation}

\subsection{Lists}

There are at least two types of lists:
\begin{enumerate}
\item enumerate
\item itemize
\end{enumerate}

There are several types of lists:
\begin{itemize}
\item enumerate
\item itemize
\end{itemize}

\end{document}
