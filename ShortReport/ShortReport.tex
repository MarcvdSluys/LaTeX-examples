% LaTeX template for a short report

\documentclass[11pt]{article}  % Use article class with 11pt letter

\usepackage[T1]{fontenc}  % 8-bit encoding, helps hyphenation of accented characters -
                          % https://tex.stackexchange.com/a/677/42066

% Use A4 paper and set narrow margins to save paper:
%\usepackage[a4paper, twoside, top=2.0cm, left=3.0cm, bottom=2.0cm, right=2.0cm]{geometry}
\usepackage[a4paper, twoside, top=2.0cm, left=2.0cm, bottom=2.0cm, right=1.5cm]{geometry}

\usepackage[english]{babel}  % Hyphenation and more for English

\usepackage{pgf}      % Include graphics inside figures using \pgfimage
\usepackage[font=small, labelfont=bf]{caption}  % Stylize figure, table, etc. captions

\usepackage{parskip}  % Replace paragraph indentations with white lines
\usepackage[hyphens]{url}  % Take care of urls, e.g. wrapping in the Bibliography (hyphens: also break at -)

% Use the appendix package for nicer appendices:
\usepackage[toc,page]{appendix}

% Citations:
\usepackage[numbers, square]{natbib}  % Use numbered citations with square brackets
\bibliographystyle{MvdS_number_url}  % Use [1], print url = field  (plain doesn't print urls)

% \usepackage[numbib,numindex]{tocbibind}  % Add ToC, List of Figures/Tables/Code listings, Bibliography and Index to ToC
\usepackage[]{tocbibind}  % Add ToC, List of Figures/Tables/Code listings, Bibliography and Index to ToC

\DeclareRobustCommand{\BibTeX}{{\normalfont\scshape Bib\kern-0.08em \TeX}}





% Data for title page:
\title{\LaTeX\ template for a short report}
\author{Marc van der Sluys}

\begin{document}



\maketitle

\tableofcontents
%\pagebreak


\section{Introduction}
\label{sec:intro}

This document is an example short report in \LaTeX\ showing how to use tables, figures and appendices, as well
as providing example sections and references.  To keep things simple and short, I use the \emph{article} class
rather than \emph{report}.  The strategies and scenarios are described in Section~\ref{sec:scenarios}, the
outcomes of the models in Sect.\,\ref{sec:results} and we discuss these outcomes in
Sect.\,\ref{sec:conclusions}.


\section{Strategies and scenarios}
\label{sec:scenarios}

Describe your two strategies in this section.  Which major variables have you chosen in the
ETM?  Why?  Which values do you use for them?

Describe your four scenarios.  If you use a table, put a tabular environment inside a (floating) table
environment, like the example in Table~\ref{tab:example}.

\begin{table}
  \centering
  \begin{tabular}{llll}
    \textbf{Scen.} & \textbf{Focus} & \textbf{Global/local} & $\bf \Delta T$ ($^\circ$C) \\
    \textbf{A1}    &  Materialistic & globalisation         & $\sim 1.4 - 6.4$ \\
    \textbf{A2}    &  Materialistic & regionalisation       & $\sim 2.0 - 5.4$ \\
    \textbf{B1}    &  Environmental & globalisation         & $\sim 1.1 - 2.9$ \\
    \textbf{B2}    &  Environment   & regionalisation       & $\sim 1.4 - 3.8$ \\
  \end{tabular}
  \caption{An example table.  Note that the label is in the caption!
    \label{tab:example}
  }
\end{table}


\section{Results}
\label{sec:results}

What are the (most important) results of your scenarios in the target year?
How do the chosen variables affect the result for each scenario?  Compare them to each other and the
base model. How much CO$_2$ emission was prevented? What are the costs per household?  Quantify your
outcomes and support your claims with arguments.  You can see some example results in Figure~\ref{fig:SRES}.
More details can be found in Appendix~\ref{app:firstApp}.

\begin{figure}
  \centering
  \pgfimage[interpolate=true,width=0.5\textwidth]{Figures/SRES_scenarios}
  \caption{Example Figure.  Note that the label is in the caption!
    \label{fig:SRES}
  }
\end{figure}


\section{Summary, conclusions and discussion}
\label{sec:conclusions}

Explain in which scenario(s) the energy transition can take place relatively painlessly, and in which
scenario(s) this is more difficult. Are the consequences acceptable, realistic, possible, affordable, \ldots?
Which scenario do you advice? Why?  How close does it get to the 2030 goals?


\section{References}

This section is intended to show my favourite use of \BibTeX\ in \LaTeX\ (depending on the target) with a
simple example.  An example of a citation is the \emph{C Programming Language Cheat Sheet} \cite{Choudhary}.
Also, I use the gcc C compiler \cite{gcc} a lot, since it is freely available for everybody.

This is a citation to a real paper \cite{leung1980scheduling}.  And this is how I cite Wikipedia
\cite{wiki:Cron}.

Note that:
\begin{itemize}
\item I use a \BibTeX\ style (*.bst) that I hacked together from number.bst, but where I made sure URLs can be
  used.  The original was by Patrick W.\ Daly, but I can't seem to find the original and hence what exactly it
  is I changed\ldots
\item I can use more than one \BibTeX\ file (*.bib).
\item Only the entries from my \BibTeX\ files that are \emph{actually cited} are listed in the bibliography.
\item After changing citations and/or references, I (may) have to run \LaTeX, \BibTeX, \LaTeX\ and \LaTeX\
  again to get everything right.
\end{itemize}

You can usually find a the code for a \BibTeX\ entry on sites where you can find papers.  Google
Scholar\footnote{After finding a publication in Google Scholar, click on the blue double quotes below the
  entry and click BibTeX at the bottom of the popup.}  (general) and ADS (physics) are useful places to go.
You can copy the code for the \BibTeX\ entry and paste it into your \BibTeX\ file (*.bib).  See
\cite{BibTeXentries} for a longer list of sites.


\begin{appendices}  %  iso \appendix
  \section{First appendix}
  \label{app:firstApp}
  
  \subsection*{With a subsection}
  
  The appendices look like simple sections\ldots\
  They offer room to many tables and figures.
  
  
  % Second appendix:
  \section{Second appendix}
  Here's a reference to Appendix~\ref{app:firstApp}.
  
\end{appendices}


\bibliography{myReferences,otherReferences}

\end{document}
